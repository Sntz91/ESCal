\subsection{Finding the Camera's Pose}
As stated in Section \ref{sec:solution}, we determined the camera's 
intrinsics in advance using Zhang's \cite{zhang2000} method. Moreover, 
the dataset contains at least four marked point correspondences between each 
perspective view and the top view. Accordingly, we determined the 
homography matrix $\mathrm{H}_p$ for every camera. Consequently, we can 
calculate every camera's location and rotation using Equation 
(\ref{eq:pose_background}).
Table \ref{tab:camera_pose} compares the logged results with our predictions.

\begin{table}[hb]
	\caption{Accuracy of the estimated camera poses}
		\begin{tabular*}{\columnwidth}{@{\extracolsep{\fill}} l cccc}
			\toprule
			&
			\multicolumn{2}{l}{\textbf{Estimated by ESCal}} & 
			\multicolumn{2}{l}{\textbf{Measured by UAV}} \\
			& Height [m] & Pitch [deg] & Height [m] & Pitch [deg] \\
			\midrule
			IMG\_01 & 10.0 & -40.5 & 10.3 & -43.0 \\
			IMG\_02 & 10.1 & -41.0 & 10.3 & -43.0 \\
			IMG\_03 & 10.2 & -41.7 & 10.3 & -43.0 \\
			IMG\_04 & 10.3 & -41.8 & 10.4 & -43.0 \\
			IMG\_05 & 10.0 & -36.4 & 10.3 & -38.1 \\
			IMG\_06 & 2.6  & -24.7 & 3.2  & -27.8 \\
			IMG\_07 & 10.3 & -34.9 & 10.3 & -38.5 \\
			IMG\_08 & 2.8  & -29.9 & 2.8  & -32.1 \\
			IMG\_09 & 9.8  & -41.0 & 10.1 & -43.5 \\
			IMG\_10 & 7.7  & -36.2 & 7.9  & -38.7 \\
			IMG\_11 & 7.7  & -44.4 & 7.9  & -47.3 \\
			IMG\_12 & 10.3 & -34.7 & 10.3 & -36.8 \\
			\bottomrule
		\end{tabular*}
	\label{tab:camera_pose}
\end{table}

This leads to an average rotational error of $2.4$ degrees. However, 
several measurement errors contribute to this result. For example, the ground 
truth values from the drone's logs contain inherent measurement inaccuracies. 
Additionally, wind affects the drone's pose, violating our assumption of an 
exact $-90$-degree pitch in our top view. In fact, we observe a 
systematic error of approximately $+2$ degrees. 
Figure \ref{fig:positions} shows 
the estimated positions for the twelve cameras within the top view.

