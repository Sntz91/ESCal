\subsection{Best Practices} 
Following the above experimental studies, we recommend the following 
best practices for using homographies under real world conditions:
\begin{itemize}
	\item Avoid collinear points.
	\item Spread reference points widely and capture many characteristics of the scene.
	\item Use more than four reference points, although four can be enough.
	\item Avoid low pixel densities and perspectives: Place the 
		cameras high, with steep angles.
	\item When using automatic methods and unable to ensure high-quality reference point pairs, use RANSAC.
\end{itemize}

\section{Conclusion}
\label{sec:conclusion}
In this work, we proposed ESCal, a method for quickly determining the pose of all 
cameras within a CN and evaluating the network's effective coverage. 
Hereby, we demonstrated that homographies, although assuming planar 
scenes, are applicable in real-world environments\textemdash provided some 
countermeasures we defined in this paper. With the right prerequisites, 
we achieved an accuracy of approximately $6$ cm with low-cost cameras. 
Higher camera resolutions further reduce this error. For example, 
this calibration technique enabled an application for localizing 
and tracking road users within a traffic scene during the 
5-Safe research project in Landshut.

Sufficient accuracy, ease of use, time efficiency, and scalability while 
not requiring overlapping fields of view, underline ESCal's potential 
for practical applications, particularly in urban settings. 
Our proposed best practices further simplify the application of our 
approach in various real-world scenes and application scenarios. 

In the future, we plan to automatically detect the reference point pairs 
within the top and perspective views. Additionally, it might be possible 
to specify the effective field of view automatically, e.g., by using 
semantic segmentation.
