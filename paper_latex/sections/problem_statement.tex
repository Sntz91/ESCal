\section{Problem Statement}
\label{sec:problem}
CNs collect data for a wide variety of applications. However, we 
must calibrate them carefully to ensure accurate data, especially for 
large-scale environments. Although several techniques exist for 
finding the extrinsic parameters of a single camera, they may not be 
practical for calibrating whole CNs. In the following, we summarize 
the key requirements for calibrating a CN.

\paragraph{Real-World Environments} Applying certain methods in real-world 
scenarios can be challenging. For instance, it may be impossible 
to interfere with ongoing traffic in urban environments. Similarly, 
setting up precise measurements could be too expensive or impractical.

\paragraph{Scalability} Calibrating large CNs requires scalable 
solutions. The approach should be quick, easy to use, and sufficiently
accurate. Furthermore, it should allow for the future addition of sensors.

\paragraph{Camera Positioning} Finally, a reliable model of the real 
world requires appropriate camera positioning. Therefore, it is 
essential to evaluate the coverage of a CN. It is important to 
know the CN's effective coverage and identify possible blind spots.
